\documentclass{article} % For LaTeX2e
\usepackage{nips14submit_e,times}
\usepackage{amsmath}
\usepackage{amsthm}
\usepackage{amssymb}
\usepackage{mathtools}
\usepackage{hyperref}
\usepackage{url}
\usepackage{algorithm}
\usepackage[noend]{algpseudocode}
%\documentstyle[nips14submit_09,times,art10]{article} % For LaTeX 2.09

\usepackage{graphicx}
\usepackage{caption}
\usepackage{subcaption}

\def\eQb#1\eQe{\begin{eqnarray*}#1\end{eqnarray*}}
\def\eQnb#1\eQne{\begin{eqnarray}#1\end{eqnarray}}
\providecommand{\e}[1]{\ensuremath{\times 10^{#1}}}
\providecommand{\pb}[0]{\pagebreak}
\DeclarePairedDelimiter\ceil{\lceil}{\rceil}
\DeclarePairedDelimiter\floor{\lfloor}{\rfloor}

\newcommand{\E}{\mathrm{E}}
\newcommand{\Var}{\mathrm{Var}}
\newcommand{\Cov}{\mathrm{Cov}}

\def\Qb#1\Qe{\begin{question}#1\end{question}}
\def\Sb#1\Se{\begin{solution}#1\end{solution}}

\newenvironment{claim}[1]{\par\noindent\underline{Claim:}\space#1}{}
\newtheoremstyle{quest}{\topsep}{\topsep}{}{}{\bfseries}{}{ }{\thmname{#1}\thmnote{ #3}.}
\theoremstyle{quest}
\newtheorem*{definition}{Definition}
\newtheorem*{theorem}{Theorem}
\newtheorem*{lemma}{Lemma}
\newtheorem*{question}{Question}
\newtheorem*{preposition}{Preposition}
\newtheorem*{exercise}{Exercise}
\newtheorem*{challengeproblem}{Challenge Problem}
\newtheorem*{solution}{Solution}
\newtheorem*{remark}{Remark}
\usepackage{verbatimbox}
\usepackage{listings}
\usepackage{mathrsfs}
\title{DiffGeoI: \\
Problem Set I}


\author{
Youngduck Choi \\
CIMS \\
New York University\\
\texttt{yc1104@nyu.edu} \\
}


% The \author macro works with any number of authors. There are two commands
% used to separate the names and addresses of multiple authors: \And and \AND.
%
% Using \And between authors leaves it to \LaTeX{} to determine where to break
% the lines. Using \AND forces a linebreak at that point. So, if \LaTeX{}
% puts 3 of 4 authors names on the first line, and the last on the second
% line, try using \AND instead of \And before the third author name.

\newcommand{\fix}{\marginpar{FIX}}
\newcommand{\new}{\marginpar{NEW}}

\nipsfinalcopy % Uncomment for camera-ready version

\begin{document}


\maketitle

\begin{abstract}
This work contains solutions to the exercises of the problem set I.
\end{abstract}

\bigskip

\begin{question}[1]
\hfill
\begin{figure}[h!]
  \centering
    \includegraphics[width=0.7\textwidth]{DG-e1-p1.png}
\end{figure}
\end{question}
\begin{solution} \hfill \\
For any topological space, path-connected implies connected. We prove the converse.
Suppose $X$ is connected. Let $x \in X$, and set
\eQb
U &=& \{ y \in X \>|\> \text{ there is a path bewteen}\> x \>\text{and}\> y\}.
\eQe
Observe that $x \in U$ so $U$ is non-empty. Now, as $X$ is connected,
if $U$ and $U^c$ are both open, then $U^c$ is empty, and $U = X$, so
$X$ is path-connected. We show that $U$ is open. Let $y \in U$. Then,
there exists an open nbd of $y$, $O$, such that $O$ is homeomorphic to 
an open ball in $R^n$(this is equivalent to the locally euclidean
condition of topological manifold). Since
path-connectedness is preserved through homeomorphism, we conclude that $O$
is path connected and $O \subset U$. Therefore, $U$ is open and similarly
$U^c$ is open, and we are done. \hfill $\qed$ 

\end{solution}

\newpage

\begin{question}[2]
\hfill
\begin{figure}[h!]
  \centering
    \includegraphics[width=0.7\textwidth]{DG-e1-p2.png}
\end{figure}
\end{question}
\begin{solution} \hfill \\
\textbf{(a)} Let $1 \leq i < j \leq n$. By symmetry, it suffices to show that
\eQb
\phi_j \circ \phi_i^{-1} : \phi_i(U_i \cap U_j) \to \phi_{j}(U_i \cap U_j) 
\eQe
is smooth. For any $a = (\dfrac{x_1}{x_i},...,\dfrac{x_{i+1}}{x_i},...,
\dfrac{x_{n+1}}{x_i}) \in \phi_i(U_i \cap U_j)$ with $x_i , x_j \neq 0$, 
\eQb
a \mapsto_{\phi_i^{-1}} [x_1,...,x_{n+1}] \mapsto_{\phi_j} (\dfrac{x_1}{
x_j},...,\dfrac{x_{j-1}}{x_j},\dfrac{x_{j+1}}{x_j},...,\dfrac{x_{n+1}}{x_i}).
\eQe
Since each coordinate map is smooth, we see that the transition map is smooth
and since the indices were arbitrary, the atlas is smooth. \hfill $\qed$

\bigskip

\textbf{(b)} Here, we choose to work with the stereographic projection chart on
$S^1$. Consider the map $\pi:S^1 \to \mathbb{R}P1$ defined by
\eQb
(x,y) &\mapsto& [1-y:x]
\eQe
for $(x,y) \in S^1$ such that $y \neq 1$, and
\eQb
(x,y) &\mapsto& [0:1]
\eQe
for $(x,y) \in S^1$ such that $y = 1$. Now, in chosen coordinates
\eQb
\pi_{1,1}(p) = p
\pi_{2,1}(p) = \dfrac{1}{p}
\pi_{1,2}(p) = \dfrac{1}{p}
\pi_{2,2}(p) = p.
\eQe
Hence, all coordinate expressions are smooth, so $\pi$ is smooth. $\pi$
is one-to-one and in local charts the inverses are smooth as well. Hence,
we see that $\pi$ is a diffeomorphism. 

\bigskip

\textbf{(c)} 
We choose the smooth structure of $S^2$ to be the one
that contains the projection charts.
Let $p = (x^*,y^*,z^*) \in S^2$ and assume without loss of generality that
$z^* > 0$. Choose a chart $(U_z^+, \psi_z)$ of $p$ where 
\eQb
U_{z^+} &=& \{ p = (x,y,z) \in S^2 \> :\> z > 0 \} 
\eQe 
with $\psi_z: U_{z^+} \to \mathbb{R}^2$ defined by
\eQb
p = (x,y,z) \in U_{z^+} \mapsto (x,y). 
\eQe
Now, observe that with $z^* > 0$, $\pi(z^*) \in U_3$, so we can choose 
$(U_3,\phi_3)$
for a chart at $\pi(z^*)$. We now claim that
\eQb
\phi_3 \circ  \pi \circ \psi_z^{-1}:\psi_z(U_z^+) \to U_3 
\eQe
is smooth at $\psi(p)$. For each $(x,y) \in \psi(U_z^{+})$, we have
\eQb
(x,y) &\mapsto_{\psi_{z}^{-1}}& (x,y,(1-x^2-y^2)^{\frac{1}{2}})
\mapsto_{\pi} [x,y,(1-x^2-y^2)^{\frac{1}{2}}] \\
&\mapsto_{\phi_3}& 
(\dfrac{x}{(1-x^2-y^2)^{\frac{1}{2}}},\dfrac{y}{(1-x^2-y^2)^{\frac{1}{2}}}) 
\eQe
which simplifies to
\eQb
(x,y) \mapsto_{\phi_3 \circ \pi \circ \psi_{z}^{-1}} (\dfrac{x}{(1-x^2-y^2)^{
\frac{1}{2}}},
\dfrac{y}{(1-x^2-y^2)^{\frac{1}{2}}}).
\eQe
Therefore, we see that each component is smooth, and $\phi^3 \circ \pi 
\circ \psi_{z}^{-1}$ is smooth, so $\pi$ is smooth.

\bigskip
 
Now, it is easy to see that $\pi$ in fact gives the diffeomorphism as claimed.
By symmetry, on any chart, $\pi$ is a bijection from $U$ to $\pi(U)$, as we have 
eliminated the $2-1$ mapping by only considering positive or negative parts
in all dimensions. It is also true that $\pi(U)$ is open. Restricted
to $\pi(U)$ by the same computation as above, $\pi^{-1}$ is smooth, so we see
that $\pi$ is a local diffeomorphism. Formally, fix $p \in S^2$. Choose a chart
that contains $p$, $U$, then, $\pi$ restircted to $U$ gives a diffeomorphism
as required. \hfill $\qed$ 

\end{solution}

\newpage

\begin{question}[3]
\hfill
\begin{figure}[h!]
  \centering
    \includegraphics[width=0.7\textwidth]{DG-e1-p3.png}
\end{figure}
\end{question}
\begin{solution} \hfill \\
This statement corresponds to 
\end{solution}

\end{document}
